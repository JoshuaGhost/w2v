\section{Experiment results} \label{experiment_results}
In this paper, following the advice in \cite{levy2015improving}, hyper-parameters are chosen as \verb|win=10|, \verb|neg=5|, \verb|dim=500| and \verb|iter=5|. English wiki dump (enwiki-latest-pages-articles on March 2nd, 2016\footnote{from https://dumps.wikimedia.org/enwiki/latest/enwiki-latest-pages-articles.xml.bz2}) is used as corpus. In table \ref{table:benchmark} the line "base-line" represents the performance of model who uses whole corpus with out any division or combination. each column stands for a evaluation dataset or training time. This benchmark is also kept the same with \cite{levy2015improving}. All of the experiments run on 18 cores of Intel(R) Xeon(R) CPU E5645 @ 2.40GHz with 126G of RAM. The size of entire corpus is approximately 14G. However because of common occupation of the server among institute, the training times are for reference only.

\begin{table*}
\caption{Performance and training time of different combination strategies}
\begin{tabular}{c|cccccccc|c}
\hline
Dataset   & AP   & MEN  & MTurk & WS353 & WS353R & WS353S & Google & MSR  & training time\ \tabularnewline \hline
Base-line & .595 & .736 & .694  & .611  & .514   & .754   & .661   & .440 & 5d 3h 11m 25s\\ \hline
\end{tabular}
\label{table:benchmark}
\end{table*}

From the result we can draw some conclusions. First of all is that $sample method$. Secondly pure PCA, especially cooperate with $order$, performs the best. This is mainly because that $reason$. If PCAs are too time consuming, a more efficient but a little bit less performing alternative can be $second alternative$. This configuration can get such nice score is mainly because $reason$. Furthermore, some by-product of the experiments are also interesting, for example $by-products$
