\documentclass[12pt, a4paper, twoside, openright]{report}

\usepackage[latin1]{inputenc}
\usepackage{graphicx}
\usepackage{ae}
\usepackage{typearea}
\usepackage[ngerman, english]{babel}

\usepackage[dvipsnames]{xcolor}

\usepackage{tikz}
\usepackage{tkz-graph}
\usetikzlibrary{backgrounds}
\usetikzlibrary{patterns}


\usepackage{subcaption}
\usepackage{caption}
\usepackage{framed}
\captionsetup[subfigure]{labelformat=empty}

\usepackage[shortlabels]{enumitem}

\usepackage{amsmath}
\usepackage{amssymb}
\usepackage{amsthm}
\usepackage{pdflscape}
\typearea{12}
\usepackage[a4paper,inner=3.5cm,outer=2.5cm,top=2.5cm,bottom=2.5cm,pdftex]{geometry}




\usepackage{subcaption}
\usepackage[export]{adjustbox}% http://ctan.org/pkg/adjustbox


\usepackage{csvsimple}

\usepackage[chapter]{algorithm}
\usepackage{float}

\usepackage{siunitx, booktabs}

\usepackage[noend]{algpseudocode}

\theoremstyle{definition}


\renewcommand{\familydefault}{\sfdefault}

% Literaturverzeichnis

\usepackage{filecontents}
\renewcommand\bibname{Bibliography}

\usepackage{afterpage}

\usepackage[hidelinks]{hyperref}


\usepackage{pgfplots}




% argument #1: any options
\newenvironment{customlegend}[1][]{%
	\begingroup
	% inits/clears the lists (which might be populated from previous
	% axes):
	\csname pgfplots@init@cleared@structures\endcsname
	\pgfplotsset{#1, legend cell align=left}%
}{%
% draws the legend:
	\csname pgfplots@createlegend\endcsname
	\endgroup
}%

\newenvironment{nstabbing}
{\setlength{\topsep}{0pt}%
	\setlength{\partopsep}{0pt}%
	\tabbing}
{\endtabbing}

% makes \addlegendimage available (typically only available within an
% axis environment):
\def\addlegendimage{\csname pgfplots@addlegendimage\endcsname}

\newcommand{\argmin}[1]{\underset{#1}{\operatorname{arg}\,\operatorname{min}}\;}

\newtheoremstyle{named}{}{}{\itshape}{}{\bfseries}{:}{\newline}{\thmname{#1} \thmnumber{#2} (#3)}
\theoremstyle{named}
\newtheorem{lemma}{Lemma}[chapter]

\newtheoremstyle{def}{}{}{}{}{\bfseries}{:}{\newline}{\thmname{#1} \thmnumber{#2}}
\theoremstyle{def}
\newtheorem{exmp}{Example}[chapter]

\newtheorem{definition}{Definition}[chapter]


\date{November 17th 2016}

\newcommand{\blankpage}{
	\newpage
	\thispagestyle{empty}
	\mbox{}
	\newpage
}


\begin{document}


\parindent 0pt
\parskip 6pt

\title{
\begin{figure}[h!]
	\includegraphics[width=0.6\textwidth, valign=t]{Grafiken/luh_logo_rgb.jpg}
	\hfill
	\includegraphics[width=0.22\textwidth, valign=t]{Grafiken/l3slogo.png}
\end{figure}
\LARGE{\textbf{Scalable Approaches for Learning Word Representations}}
\\[2cm]
\normalsize{\textbf{Master Thesis}}\\
\normalsize{Master of Science in Informatics}
\\[2cm]
\normalsize{\textbf{Zhang, Zijian}}\\
\normalsize{Matriculation Number: 3184680}
\\[2cm]
\normalsize{First Examiner: Prof. Dr. Avishek Anand}\\
\normalsize{Second Examiner: Prof. Dr. techn. Dipl.-Ing. Wolfgang Nejdl }\\
\normalsize{Advisor: XXXXXXXXX}
}
\author{}
\maketitle

%\newpage
%\setcounter{page}{2}
%\thispagestyle{empty}~
%\newpage

%\begin{abstract}
%\setcounter{page}{3}
%\input{00_abstract_de}
%\end{abstract}

%\newpage
%\setcounter{page}{4}
%\thispagestyle{empty}~
%\newpage

%\setcounter{page}{5}


\pagenumbering{Roman}
\tableofcontents

\parskip 12pt



\parskip 6pt

%Buecher, die nicht zitiert werden ins Literaturverzeichnis
\nocite{*}
%Literarut-/Tabellen-/Abbildungsverzeichnis
%\listoffigures
%\listoftables
\addcontentsline{toc}{chapter}{Bibliography}
\bibliography{literatur}

\addcontentsline{toc}{chapter}{List of Figures}
\listoffigures
\addcontentsline{toc}{chapter}{List of Tables}
\listoftables



\end{document}
